\section{Introduction}
The past couple of decades has been marked by economic transformation, demographic change, declining fertility, technological progress, advancement in education and health, and shifts in social norms that has in many ways helped women enter the labour force around the world.  Yet, despite a backdrop of economic, social, and political transformation, the gender gap in labour force participation continues to persist at 27 percentage points and the participation rates of women has endured a slowdown in the past decade. This suggests that globally, we are still far away from closing the gender gap as women continue to face a multiplicity of constraints restricting their capabilities and freedom to access the labour market.  

A recent joint ILO-Gallup survey demonstrates that women are clearly facing multiple constraints when the majority of women (70 per cent) prefer to work a paid job while approximately half of women globally still remain out of the labour force. Even among those out of the labour force, their preference to work remains a majority at 58 per cent. Ensuring that all women who have the preference to work can freely do so, equally as men, increases their autonomy in the household and in the public domain while contributing to other dimensions of gender equality (Kabeer 2002, 2004; Elson 1999). Yet, not all decisions to participate are freely made when women under economic constraints have no choice, but to work out of necessity, while often carrying the double burden of both unpaid and paid labour. Moreover, increased participation does not necessarily imply that women are better off considering that they are overrepresented in low-paid and labour-intensive sectors and occupations with sub-optimal working conditions, which may simultaneously discourage participation (Kucera and Tejani, 2014, Cagatay and Ozler 1995). 

Alternatively, for certain groups of women, the trade-off of participation for higher education or retirement can be a more desirable and preferred opportunity. Hence, women around the world not only face various drivers and constraints to participation but, how their preference and decision to participate are affected depends on their position in their life-cycle (\cite{besamusca2015working}). For instance, prime-age working women often face greater care demands when married and with younger children than youth or older women.

As a result of the findings of the recent ILO-Gallup survey, on women's preferences and their life evaluations associated with the labour market, this paper contributes to the ongoing discussion of female labour force participation by analysing which factors constrain or drive participation and to what extent in the life-cycle. The paper expands previous work in two important dimensions. First, it utilizes the results from newly developed survey questions by the International Labour Organization (ILO) designed to specifically identify personal preferences of women, societal and household acceptability of women working, and the life-evaluation of women in the labour market. This motivates the paper to address the following questions: (i) to what extent is the decision to participate driven by individual preferences? (ii) how do gender roles shaped by social norms affect participation? (iii) to what extent do socio-economic constraints affect participation? and (iv) how do these socio-economic constraints differ according to the life-cycle? Second, the ILO survey questions administered within the 2016 Gallup World Poll, creates an immensely rich dataset covering 149,000 adults in 142 countries, thereby representing 99 per cent of the global adult population and providing the opportunity to study the labour participation of women across all income and cultural groups. Moreover, this paper employs a life-cycle analysis by distinguishing different age groups (youth, prime-age working, 55+) to account for the life-cycle effects on women's participation to consider the different drivers or constraints women face at different points in their life. 

In Section~\ref{sec:framework}, we review the relevant literature and construct a conceptual framework of the fundamental drivers of female labour force participation.  Section~\ref{sec:data} summarizes the dataset and Section~\ref{sec:methodology} presents the empirical methodology. Section~\ref{sec:results} reports the results. Section~\ref{sec:conclusion} concludes and discusses the  potential research subsequently arising from the paper's findings.
 