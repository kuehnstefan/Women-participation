\section{Conclusion}\label{sec:conclusion}

In conclusion, this paper provide strong evidence that a woman's preference to work has a significant positive effect on participation. However, preferences are a function of a range of factors, including women's life-cycle circumstances, their socio-economic conditions, their gender roles and the conditions of the local labour market of the region. Additionally, a number of socio-economic constraints also influence the probability to participate. Importantly, the influence of these constraints is not the same for all women, but varies according to the life cycle and to country characteristics. This has important policy implications, as it is necessary to know the constraints that women face before designing policies aiming at reducing gender gaps in the labour market.