\section{Data description}\label{sec:data}

The principal data source is the 2016 instalment of the Gallup World Poll. This is a survey undertaken by Gallup, where 149 thousand persons in 142 countries have been asked an extensive set of questions on a whole range of issues. Table~\ref{tab:vardescription} presents the questions that have proven to be of relevance for explaining female labour force participation. Due to missing data in some countries, the sample is reduced to 60126 women in 121 countries.

Importantly, no observations are dropped within countries that form part of the analysis, the sample reduction affects complete countries. Hence, the estimated results are unbiased for the countries under analysis. Given that 121 countries is still a very large sample, and that missing countries tend to be rather small, obtained results can be generalized.


\begin{longtable}{|p{0.3\textwidth}|p{0.1\textwidth}|p{0.6\textwidth}|}
%	\centering
	\caption{Description of variables}\label{tab:data_descr}\\
	\hline
		\textbf{List of variables}& \textbf{Symbol} & \textbf{Description} \\
		\hline
		\endfirsthead
	\caption{continued...}\\
		\hline
		\textbf{List of variables} & \textbf{Symbol} & \textbf{Description}  \\
		\hline
		\endhead
		\hline
		\endfoot
		Age &$AGE$ & Young (15-24) \\
		&& Prime age (25-54) \\
		&& Old (55+)\\
		\hline
		Children &$CHD$ & Dummy if having at least one child\\
		\hline
		Household members &$HHM$ & Actual numeric value \\
		\hline
		Education &$EDU$  & Primary education (base level) \\
		&& Secondary education \\
		&& Tertiary education \\
		\hline
		Relationship &$RLS$ &Single, widowed, separated or divorced (base level) \\
		&& Married, living with partner\\
		\hline
		Internet, Phone, Communication & $INT$, $PHO$, $COM$ & Dummy for having access to internet or phone (mobile or landline), or both \\
		\hline
		Urban &$URB$ & Rural (base level), urban \\
		\hline
		Poverty &$PVT$ & No poverty (base level)\\
		&& Mild poverty (not enough money for either food or shelter) \\
		&& Severe poverty (not enough money for food and shelter) \\
		\hline
		Religion &$REL$ & Secular/atheist (base level) \\
		&& Other \\
		&& Christian \\
		&& Muslim \\
		&& Hindu \\
		\hline
		\multirow{1}{0.3\textwidth}{Question "Do you prefer to work in a paid job, stay at home, or do both"} &$PFW$& Response: \\
		&& "Stay at home" (base level)\\
		&& "prefer paid job" or "both" \\
		\hline
		Acceptability of paid work &$ACW$ & Dummy, women answering that household members find it acceptable for a women to work in a paid job. \\
		\hline
		Opportunity (better, worse) &$OPP$ & Dummy when person finds that women have better (worse) opportunity on labour market than men given equal education, base level same opportunity. \\
		\hline
		Job climate index &$JCL$ & Continuous, an index (values 0, 50, 100) computed by Gallup capturing the perceived labour market environment by a respondent.  \\
		\hline
		Roads &$RDS$ & Dummy, whether respondents are satisfied (1) or dissatisfied with the road infrastructure in their environment\\
		\hline
		Law and order &$LAW$ & Continuous, an index (7 levels) from Gallup capturing how respondents view the state of law and order.\\
		\hline
		\multirow{1}{0.3\textwidth}{Major challenge faced in labour market:} &$CHG$ & Lack of flexible work hours (base level)\\
		&& Balance between work and family or home\\
		&& Lack of affordable care for children or relatives\\
		&& Family members don't approve of women working\\
		&& unfair treatment at work/abuse/harassment/discrimination\\
		&& Lack of good-paying jobs\\
		&& Unequal pay for doing similar work as men\\
		&& Lack of transportation/lack of safe transportation \\
		&& People prefer to hire or promote men\\
		&& Lack of skills, experience or education
	\label{tab:vardescription}%
	\end{longtable}%

The survey contains basic information about respondents such as their age, sex, family status, education level and religion. Next, some questions cover the labour market status of respondents. Furthermore, questions relate to the environment in which subjects live, revealing for example their household size, the number of children in the household, whether they live in an urban or rural environment and the availability of communication devices in the home. Frequently, the survey also contains questions on household income. However, the 2016 instalment did not cover this, containing only questions on whether households lack money for food or shelter, respectively, which serves as an indicator of poverty. 

Also, the survey contains a large number of questions about the opinion of respondents on certain issues, such as how favourable they see the United States. The most interesting of these opinion questions for the purposes of this analysis is the job climate index, which shows how respondents evaluate the current prospects for jobs in their environment.

Finally, the 2016 instalment of the Gallup World Poll covers a number of questions designed by the ILO in order to identify the opinion of women and men about women's position in the labour market. A detailed analysis of these opinions has been published in \citet{ilogallup2017}. Four of these questions are of importance for this paper. 

The first one asks women \enquote{Would you PREFER to work at a paid job, or stay at home and take care of your family?} The choices were either \enquote{Work at paid job}, \enquote{Stay at home} or \enquote{Both}. The second question asked \enquote{It is perfectly acceptable for any woman in your family to have a paid job outside the home IF SHE WANTS ONE. Do you agree?} The answering choices were \enquote{Agree} or \enquote{Disagree}.

The third question asked \enquote{Please think about women who work at paid jobs in [country/territory name] today. What do you think is the BIGGEST challenge these women face?} Responses were unprompted, and then coded into one of ten categories (see table~\ref{tab:vardescription}). The fourth question asked was \enquote{If a woman has similar education and experience to a man, does she have a better opportunity, the same opportunity, or a worse opportunity to find a good job in the city or area where you live?} All of the questions could also be answered with \enquote{Don't know}.

Country-level information is retrieved from other sources. The labour force participation rates for women are taken from ILOSTAT. The GDP per capita, as well as the country classification into income categories, is taken from the World Bank's World Development Indicators.

